\documentclass[norsk]{beamer}

\title{En verden av polynomer}
\author{Jon-Magnus Rosenblad}
\date{13. Mars, 2024}

\begin{document}

\frame{\titlepage}

\begin{frame}{Polynomer}
    Hva er et polynom?
    \begin{definition}
        Et \textit{polynom} er et uttrykk på formen
        \[
            a_0 + a_1 x + a_2 x^2 + \dots + a_n x^n
        \]
        hvor $a_0, a_1, \dots, a_n$ er skalarer (varierer ikke med $x$).

        Verdiene $a_0,\dots, a_n$ kalles \textit{koeffisientene} til polynomet,
        og tar ofte verdi blant de reelle tallene $\mathbb R$,
        men kan også være rasjonale tall $\mathbb Q$,
        heltall $\mathbb Z$,
        eller komplekse tall $\mathbb C$,
        (eller noe helt annet).

        Den største mulige verdien for $n$ slik at $a_n \neq 0$ kalles
        \textit{graden} på polynomet.

        Et polynom kalles \textit{monisk} om $a_n = 1$.

        Symbolet $x$ kalles \textit{indeterminanten} til polynomet.

        Mengden av polynomer med Reelle koeffisienter med indeterminant $x$
        benevnes $\mathbb R[x]$.
    \end{definition}
\end{frame}

\begin{frame}{Algebraens fundamentalteorem}
    \begin{theorem}
        Et polynom $p$ av grad $n$ har ikke flere enn $n$ røtter.
    \end{theorem}
\end{frame}

\begin{frame}{Polynomer over $\mathbb C$}
    \begin{definition}
        De \textit{komplekse tallene} $\mathbb C$ er mengden av
        alle mulige røtter av polynomer med koeffisienter i
        de reelle tallene $\mathbb R$.
    \end{definition}
\end{frame}

\begin{frame}{Algebraens fundamentalteorem over $\mathbb C$}
    \begin{theorem}
        Et polynom $p$ av grad $n$ med komplekse koeffisienter kan faktoriseres
        på formen
        \[
            a_n (x - x_1)\dots(x - x_n).
        \]
        En slik faktorisering er unik opp til permutasjon av faktorene.
    \end{theorem}
    \begin{corollary}
        Et polynom $p$ av grad $n$ med reelle koeffisienter kan
        faktoriseres på formen
        \[
            p = p_1\dots p_m
        \]
        hvor $p_1,\dots, p_m$ er polynomer av grad høyst $2$.
    \end{corollary}
\end{frame}

\begin{frame}{Hvordan finne polynomrøtter?}
    \begin{lemma}
        Et polynom på formen $ax^2 + bx + c$
        har røtter gitt ved
        \[
            \frac {
                -b \pm \sqrt{b^2 - 4ac}
            }{2a}
        \]
    \end{lemma}
\end{frame}

\begin{frame}{Polynomer i flere variable}
    \begin{definition}
        Et \textit{polynom i $0$ variabler} er en skalar $a_0$.

        Et \textit{polynom i $m$ variabler $x_1,\dots,x_m$} er et uttrykk på
        formen
        \[
            p_0 + p_1 x_m + p_2 x_m^2 + \dots + p_n x_m^n
        \]
        hvor $p_0,\dots, p_n$ er polynomer i $m - 1$ variabler
        $x_1,\dots, x_{m - 1}$.
    \end{definition}
    \begin{example}
        \begin{itemize}
            \item $x^2 + y^2 - 9$
            \item $x_1^2 + x_2^2 - 9$ er det samme polynomet,
                men indeterminantene er gitt ved andre symboler.
        \end{itemize}
    \end{example}
\end{frame}

\begin{frame}{Graden av et polynom flere variable}
    \begin{definition}
        Et \textit{monom} er et polynom som kan beskrives uten addisjon,
        dvs.\ på formen $a x_1^{r_1}\dots x_m^{r_m}$.

        Summen av eksponentene $r_1 + \dots + r_m$ er \textit{graden til $p$}.
    \end{definition}
    Alle polynomer kan unikt skrives som en sum av monomer av forskjellig grad.
    Graden til polynomet bestemmes av monomet av høyest grad.
\end{frame}

\begin{frame}{Implisitte kurver og parametrisering}
    \begin{definition}
        En \textit{implisitt kurve $C$ i planet $\mathbb R^2$}
        er en mengde på formen
        \[
            C = \{ (x,y)\in \mathbb R^2 \mid p(x,y) = 0\},
        \]
        hvor $p$ er et polynom,
        dvs. $C$ er mengden av løsninger til likningen $p(x,y) = 0$.
    \end{definition}
\end{frame}

\begin{frame}{Rasjonale funksjoner}
    Merk at ved å tillate rasjonale funksjoner for å definere en kurve
    får vi ikke flere kurver enn om vi bare tillater polynomer.
\end{frame}

\begin{frame}{Geometri og nullpunktsmengder}
    ``Hvordan ser nullpunktsmengden til polynomet $p(x)$ ut?''
    er det samme som å spørre ``Hva er røttene til $p(x)$?''
    Tilsvarende, kan vi finne
    \begin{example}[Geometri i $1$ dimensjon]
        Merk at om $A = \{ p(x) = 0 \}$
        og $B = \{ q(x) = 0\}$
        er to implisitte mengder av punkter,
        så er $A\cup B = \{ p(x) q(x) = 0 \}$.
    \end{example}
    \begin{corollary}
        Alle endelige mengder av punkter på talllinja kan
        beskrives implisitt.
    \end{corollary}
    \begin{example}[Ikke geoemtriske kurver]
        Grafen til funksjonen $f(x) = e^x$ kan ikke beskrives
        som en implisitt kurve av polynomer.
    \end{example}
\end{frame}

\begin{frame}{Ikkeparametriserbare kurver}
    \begin{example}
        Kurven $\{ y^2 = x^3 - x\}$ kan ikke parametriseres.
    \end{example}
    \begin{theorem}[Invers funksjonsteorem]
        Alle implisitte kurver har en \underline{lokal}
        parametrisering.
    \end{theorem}
\end{frame}

\begin{frame}{Rasjonale polynomer}
\end{frame}

\end{document}
